\chapter{萬事萬物皆物件 \\
Everything is an Object}


雖然Java 奠基於C++之上,兩相比較,Java 卻是個更「純粹」的物件導向程式語言。
C++和 Java 都是混合型程式語言 (hybrid language) 。但 Java
的設計者認為這種混合特性在 Java 中不像在 C++中那麼重要。
混合型程式語言提供多種程式設計風格; C++之所以為混合型語言,是為了回溯相容於 C。
C++ 做為 C 語言的超集 (superset),勢必將 C 語言中許多不適合出現於 C++
的特性一併囊括進來。這些性質使 C++ 在某些方面顯得過於複雜。
Java 程式語言在設計上,徹底假設使用者僅以物件導向模式來進行程式設計。
這麼一來,你便得先轉換自己的思維模式,進入物件導向的世界,才能夠開始使用
Java (除非你的思維模式早已是物件導向形式)。透過這個入門基本功夫,爾後再學習、
再使用其他的OOP 語言,上手的速度就可以更快。本章之中,我們將看到 Java
程式的基本組成,並學到一個十分重要的觀念:在 Java 程式中,萬事萬物皆物件
(object),即使 Java 程式本身,也是一個物件。
\section[Reference 是操控物件之鑰錀]{Reference 是操控物件之鑰錀 \\
You manipulate objects with references}
每個程式語言都有其獨特的資料操作方式。有時候,程式員對於正在處理的資料,
必須持續關注其型別 (type) 究竟為何。你究竟是直接操作物件,或是透過某種中介形式
(例如 C 和 C++ 的指標),因而必須以某種特殊語法來看待物件呢?
這一切在\marginpar{\fbox{102}} Java 中都大大簡化了。
所有事物都被視為物件,單一固定的語法便可通用各處。不過,
雖然抽象概念上你可以把所有東西都「視為」物件,但用以操控物件的識別字,
實際上卻只是其 ``reference'' (引用、參照、指引)
而已\footnote{這裡或許會引發爭論。有些人認為「很明確地,這就是指標
(pointer)」。但這種說法對底層實作方式進行了某種假設。事實上, Java 的
reference 在語法上比較接近 C++ 的 reference。
本書第一版中我發明了一個應用於此處的新用語 ``handle'' ,因為 C++ reference
和 Java reference 相較之下,某些地方還是有著頗為重要的差異。我那時剛離開
C++ 陣營,而且我不想對那些原本已經了解 C++的程式員帶來混淆,那些人應該會是
Java 使用者的大宗。但是在此第二版中,我決定換回最被廣泛使用的 ``reference''
一詞。那些從 C++ 陣營轉換過來的人們,理應更能妥善理解並運用這個詞彙,
無礙地游走於兩個語言之間。儘管如此,還是有人不同意這個術語。
我曾經讀過的某本書是這麼說的:「宣稱 Java 支援 pass by reference 的那些人,
根本一派胡言!事實上是徹底的 pass by value」。該作者認為, Java
的物件識別字實際上就是個 object reference,所以絕對沒有任何 pass by reference
之情事,每一樣東西都是 pass by value。也許有人會在諸如此類的爭辯上咬文嚼字,
但我想我的方法可以簡化觀念的釐清,而不傷害到任何東西。唔,
那些精明得像律師的語言學專家也許會認為我在撒謊,
但我認為我所提供的不過是適度抽象化的結果罷了。} 。物件和 reference 之間的關係,
好比電視和遙控器之間的關係一樣。只要你手上握有遙控器,便可以控制電視。
當某人想要改變頻道或是將音量關小一點時,你實際上操控的是遙控器,間接透過這個
reference 來改變實物性質。如果你想在房間裡頭走來走去,同時保有對電視的控制,
只要隨身帶著遙控器 (也就是reference),不必揹著笨重的電視。

是的,遙控器於電視機之外獨立存在。你可以擁有某個 reference,
卻不見得要將它連接至某個物件。如果你想儲存某個字或某個句子,你可以產生一份
String reference,像這樣:

\begin{Verbatim}[frame=single]
String s;
\end{Verbatim}

但由於這麼寫只會產生一個 reference,不會產生實際的物件,因此此刻將訊息傳送給 s,
執行時期便會發生錯誤。這是因為 s 並未實際連接到任何實物身上
(也就是沒有相對可控制的電視)。所以,比較安全的作法就是在每次產生
reference 的同時便加以初始化:

\begin{Verbatim}[frame=single]
String s = "asdf";
\end{Verbatim}

上述程式碼用到了\marginpar{\fbox{103}}
Java 的一個特性:透過雙引號將文字括起來,便可以對字串進行初始化。不過,
這是一種比較特殊的初始化方式。通常你得使用更一般化的方式來對物件進行初始化。
\section[所有物件都必須由你建立]{所有物件都必須由你建立 \\
You must create all the objects}
產生某個reference 之後,你多半會想把它連接到某個新產生的物件去。
通常我們會透過關鍵字 new 來做。 new 的意思是「請給我一個新物件」。
所以上述例子中,你也可以寫成:

\begin{Verbatim}[frame=single]
String s = new String("asdf");
\end{Verbatim}

這麼寫對 s 來說,不僅代表著「讓我成為一個新的 String」,也提供了用以產生 String
的原始字串。 String 並非唯一的既存型別。Java 提供了多不勝數的現成型別。
然而重要的是,你可以建立自定型別。事實上這是 Java 程式設計中最主要的一環。
透過本章接下來的篇幅,你可以學習如何建立自定型別。
\subsection{儲存在哪裡}
程式執行時究竟如何放置物件?其記憶體佈局方式如何?如果能夠了解這一點,
會帶來很大幫助。有六個地方可以存放資料:
\begin{enumerate}
\item 暫存器 (Registers)。這是速度最快的儲存場所,因為暫存器位於處理器內部,
這一點和其他種類的儲存媒介都不一樣。不過,由於暫存器個數有限,
所以編譯器會根據本身需求適當地配置暫存器來使用。作為一個程式員,
你不僅無法直接碰觸暫存器,也沒辦法在程式裡頭感覺到暫存器的任何存在跡象。
\item Stack(堆疊),位於一般的 RAM (random-access memory,隨機存取記憶體)中。
處理器經由其指標 (stack pointer)提供直接支援。當程式配置一塊新的記憶體時,
stack 指標便往後移;釋放記憶\marginpar{\fbox{104}}體時,指標則往前移回。
這種方式不僅很快,效率也高,速度僅次於暫存器。由於 Java 編譯器有責任產生「將
stack 指標前後移動」 的程式碼,所以它必須能夠完全掌握它所編譯的程式中「存在
stack 裡頭的所有資料的實際大小和存活時間」。如此一來便會限制程式的彈性。
由於這個限制,儘管我們可以將物件的 reference 儲存於
stack 內,但卻不能將一般的Java 物件也置於其上。
\item Heap(堆積)。Heap 是一種通用性質的記憶體儲存空間(也存在於 RAM 中),
用來置放所有Java 物件。Heap 勝過 stack 之處在於,編譯器不需知道究竟得從
heap 中配置多少空間,也不需知道從 heap 上配置的空間究竟需要存在多久。因此,自
heap 配置儲存空間可以獲得高度的彈性。每當你需要產生物件,只需在程式碼中使用
new,那麼當它執行的時候,便會自heap 配置空間。當然,你得為這樣的彈性付出代價:
從heap 配置空間,比從stack 配置(假設你真的可以在 Java 中像C++一樣地自
stack 產生物件的話),所耗費的時間多了不少。
\item 靜態儲存空間(Static storage)。這裡使用「靜態」一詞,指的是
「在固定位置上」(也在RAM 裡頭)。靜態儲存空間存放著「程式執行期間」
一直存在的資料。你可以使用關鍵字static,將某個物件內的特定成員設為靜態,但
Java 物件本身絕無可能置於靜態儲存空間中。
\item 常數儲存空間(Constant storage)。常數值常常會被直接置於程式碼裡頭。
因為常數是不會改變的,所以也是安全的。有時候常數會和外界隔離開來, 
所以也可以放到唯讀記憶體( read-only memory,ROM)中。
\item Non-RAM 儲存空間。如果資料完全存活於程式之外,那麼即使程式不執行,
資料也能夠繼續存在,脫離程式的控制。streamed objects
(串流化物件)和 persistent objects(永續性物件)便是主要的兩個例子。在
streamed objects 形式中,物件被轉換為一連串的 bytes。這些
bytes 通常用來傳送到另一部機器。在
persistent objects 的形式中,物件被儲存於磁碟,所以即使程式結束了,
這些物件的狀態還能夠繼續保有。此類儲存空間的特點在於,
它們能夠\marginpar{\fbox{105}}將物件轉換為可儲存於其他媒介的形式,
並在必要的時候將所儲存的資料還原成可儲存於RAM 中的一般物件。
Java 提供了所謂「輕量級永續性 (lightweight persistence)」,
新版本有可能提供更完善的解決方案。
\end{enumerate}
\subsection{特例:基本型別 (primitive types)}
在「物件型別」(object type)之外,還有一種型別(type)應該被特別對待。
你可以把這類型別視為所謂的「基本型別(primitive types)」,
它們會在接下來的程式設計過程中屢屢出現。這一類型別之所以應該受到特別待遇,
因為如果透過 new 來產生這一類極小、極簡單的變數,會因
「new 將物件置於heap 之上」而效率不彰。因此對於此類變數,Java 採取C/C++ 的方式,
也就是不以new 配置其空間,而是產生一種所謂的 ``automatic'' 變數
(不再是reference 形式),來解決效率問題(譯註:這便混入了non-OO 語言性質)。
此類變數直接存放資料值,並置於stack, 因此在空間的配置和釋放上,效率好很多。

每一種基本型別所佔的空間大小,在Java 裡頭是確切不變的。
它們的大小不會像大多數程式語言那樣「隨著機器的硬體架構而改變」。這是
Java 程式具備高度可攜性的原因之一。

\bigskip
\begin{table}[htbp]
\center
\begin{tabular}{lllll}
\toprule
基本型別 & 大小 & 最小值 & 最大值 & 外覆型別\\
\midrule
boolean & - & - & - & Boolean\\
\midrule
char & 16-bit & Unicode 0 & Unicode $2^{16}-1$ & Character \\
\midrule
byte & 8-bit & -128 & +127 & Byte \\
\midrule
short & 16-bit & $-2^{15}$ & $+2^{15}-1$ & Short \\
\midrule
int & 32-bit & $-2^{31}$ & $+2^{31}-1$ & Integer \\
\midrule
long & 64-bit & $-2^{63}$ & $+2^{63}-1$ & Long \\
\midrule
float & 32-bit & IEEE754 & IEEE754 & Float \\
\midrule
double & 64-bit & IEEE754 & IEEE754 & Double\\
\midrule
void & - & - & - & Void\\
\bottomrule
\end{tabular}
\caption{所有數值型別(numeric types)都帶有正負號。}
\end{table}

\textbf{boolean}\marginpar{\fbox{106}}
型別的容量未有明確定義;其值僅能為常數值 true 或 false。

基本型別有所謂的「外覆類別(wrapper classes)」。如果你想在
heap 內產生用以代表該基本型別的非原始物件
(nonprimitive object),那麼,外覆類別便可派上用場。例如你可以這麼寫:
\begin{Verbatim}[frame=single]
char c = 'x';
Character C = new Character(c);
\end{Verbatim}

或是這麼寫:
\begin{Verbatim}[frame=single]
Character C = new Character('x');
\end{Verbatim}

之所以要這麼寫,其原因得稍後章節才能說明清楚。
\subsection{高精度數值 (High-precision numbers)}
Java 提供了兩個用來進行高精度計算的classes : BigInteger 和
BigDecimal。雖然它們勉強可以視為外覆類別,但兩者都沒有對應的基本型別。

不過,這兩個 classes 所提供的操作方法,和基本型別所能執行的十分相像。
這句話的意思是,所有可在 int 或 float 身上做的事情,都可施行於
BigInteger 和BigDecimal 身上,只不過必須以「函式叫用方式」取代基本型別的運算子
(operators)罷了。這麼做會提高複雜度,所以運算速度會比較慢。是的,
這裡打的算盤是:以速度換取精度。

BigInteger 所提供的整數支援任意精度。
也就是說你可以很精確地表示任意長度的整數數值,不會在運算過程中喪失任何資訊。

BigDecimal 提供任意精度的定點數(fixed-point numbers);
在需要精確金融計算的場合裡,它便可以派上用場。(譯註:所謂定點數
fixedpoint numbers,是指小數點位置固定,或稱小數位數固定數。
對應的是所謂的浮點數 floating-point numbers。)

如果想知道這兩個 classes 的建構式 (constructors) 及其呼叫法,
請參閱你手邊的線上文件。

\subsection{Java 中的陣列 (array)}\marginpar{\fbox{107}}
幾乎所有程式語言都提供了陣列。在C/C++中,陣列的使用是一件危險的事情,
因為這兩個語言的陣列其實就是一塊記憶體而已。
如果某個程式存取了陣列區塊之外的位址,或在記憶體被初始化前便使用之
(這都是程式設計裡頭再常見不過的錯誤),便會導致不可預期的錯誤發生。

Java 的主要目標之一是安全。許多不斷困擾 C/C++程式員的種種問題,
Java 都不再重蹈覆轍。Java 保證陣列一定會被初始化,而且程式員對陣列的存取,
無法逾越範圍。這種「對存取範圍的檢查」所付出的代價便是:
每個陣列得額外多出一點點空間,並且得在執行時期對陣列索引值進行檢查。
Java 認為,這麼做所帶來的安全性和生產力提升,是值得的。

當你產生某個儲存物件的陣列時,真正產生的其實是個儲存 references 的陣列。
此一陣列建立之後,其中的每一個 reference 皆會被自動設為某個特殊值。該值以關鍵字
null 表示。當Java 看到null 值,便將這個 reference 視為「不指向任何物件」。
使用任何reference 之前,你必須先將某個物件指派給它。如果你使用某個
reference 而其值為null,便會在執行期發生錯誤。因此,陣列操作上的常犯錯誤,在
Java 中均可避免。

你當然也可以產生一個陣列,用來儲存基本型別。編譯器一樣保證初始化動作的必然進行:
這一次它會將陣列所佔的記憶體全部清為零。

稍後我們還會再討論陣列。

\section{你再也不需要摧毀物件 \\
You never need to destroy an object}
大多數程式語言中,變數的壽命(lifetime)觀念,佔據程式設計工作中非常重要的一部份。
變數可以存活多久?如果你認為應該摧毀某個物件,何時才是最佳時機?
圍繞在變數壽命上的種種困惑,可能形成滋生臭蟲的溫床。本節將說明
Java 如何大幅簡化這個議題:是的,它為你做掉了一切工作。

\marginpar{\fbox{108}}
\subsection{生存空間 (Scoping)}
大多數程序式(procedural)語言都有所謂「生存空間(scope)」的概念。
這個概念決定了某一範圍內的所有變數名稱的可視性(visibility)和壽命。在
C、 C++ 、 Java 之中,生存空間由兩個成對大括號決定,例如:

\begin{Verbatim}[frame=single]
{
 int x = 12;
 /* only x available */
 {
 int q = 96;
 /* both x & q available */
 }
 /* only x available */
 /* q "out of scope" */
}
\end{Verbatim}
生存空間內所定義的變數,都只能用於生存空間結束之前。
程式縮排格式,可以讓Java 程式碼更易於閱讀。由於Java 是一種自由格式
(free-form)的語言,所以不論空白、跳格、換行,都不會影響程式。
記住,即使以下寫法在C/C++中合法,在Java 裡頭卻不能這麼做:
\begin{Verbatim}[frame=single]
{
 int x = 12;
 {
 int x = 96; /* illegal */
 }
}
\end{Verbatim}
編譯器會認為變數 x 已經被定義過了。這種在
C/C++中「將較大生存空間中的變數遮蔽起來」的能力,Java 是不提供的。因為
Java 設計者認為,這麼做容易導致對程式的誤解和混淆。

\subsection{物件的生存空間 \\
Scope of objects}\marginpar{\fbox{109}}
Java 物件所擁有的壽命,和基本型別是不一樣的。當你使用new 來產生一個
Java 物件,即便離開了生存空間,該物件依然存在。因此如果你這麼寫:

\begin{Verbatim}[frame=single]
{
 String s = new String("a string");
} /* end of scope */
\end{Verbatim}

s 這個reference 將在生存空間之外消失無蹤。但是,s 先前所指的那個
String 物件,仍然會繼續佔用記憶體。如果單看上面這段程式碼,無法存取該物件,
因為唯一指向它的那個reference,已經離開了其生存空間。後繼章節中你會看到,
在程式執行過程中, 「指向物件」的那些個
references 是如何地被四處傳遞和複製。

經由new 所產生的物件,會在你還需要用到它時,繼續存在。所以許多
C++程式設計上的問題在Java 中便消失於無形了。在C++中,最困難的問題似乎肇因於,
程式員不想獲得來自語言的幫助,藉以確保他自己在需要使用物件的時候,
物件的確能夠供其使用。更重要的是,在C++裡頭,你得在用完物件之後,
確保它們千真萬確地被摧毀掉。

這麼一來便浮現了一個有趣的問題。倘若Java 讓這些物件繼續存在,
究竟是什麼機制使這些物件不會毫無節制地佔用記憶體,進而搞垮你的程式呢?這正是
C++裡頭可能發生的問題,也正是神奇之所在。Java 之中有一種所謂的「垃圾回收器
(garbage collector)」機制,它會逐一檢視所有透過
new 所產生的物件,並在這些物件不再被引用時(譯註:也就是不再有任何
reference 指向它們時),知道這一事實。然後,垃圾回收器便釋放這些物件的記憶體,
提供他用。這代表你根本不用操心記憶體回收問題,
只要關心物件的產生就好了。所有物件,在你不再需要它們的時候,都會自動消失。
這麼一來便消除了所謂的「記憶體洩漏(memory leak)」問題。
這個問題正是因為程式員忘了將記憶體釋放掉而引起的。

\section{建立新的資料型別:class}\marginpar{\fbox{110}}
如果一切都是物件,那麼究竟什麼東西用來決定某一類物件的外觀長相和行為舉措呢?
換另一種方式說,究竟是什麼制定了物件的type 呢?你也許會預期有個關鍵字type,
這才符合它的意義。不過,從歷史沿革來看,大多數物件導向程式語言都使用
class 這個關鍵字來代表「我即將告訴你,
此一新式物件的長相與外觀」。定義新的class 時,請在關鍵字class
(由於出現太過頻繁,此後不再以粗體字表示)之後緊接著新型別的名稱,例如:

\begin{Verbatim}[frame=single]
class ATypeName { /* class body goes here */ }
\end{Verbatim}

這麼做便能制定新型別,讓你得以透過new 來產生此一類型物件:

\begin{Verbatim}[frame=single]
ATypeName a = new ATypeName();
\end{Verbatim}

AtypeName class 的主體部份只有一行註解(成對的/* */ 所含括的內容即為註解。
本章稍後馬上會討論之),所以你沒辦法透過它來做些什麼事情。
事實上你得先為它定義一些函式(methods),然後才能夠告訴它如何多做一點事
(也就是說,你才能夠將有意義的訊息傳送給它)。
\subsection{資料成員(fields)和函式(methods)}
當你定義class 時(事實上在Java 裡頭你需要做的事情無非就是:定義
classes、產生物件、將訊息發送給物件),你可以將兩種成員放進去:資料成員
(data members , 有時稱為欄位, field) , 以及成員函式
(member functions),後者通常稱為methods(譯註:本書將以「函式」稱之)。
資料成員可以是任何類型的物件,只要你可以透過它的
reference 來和它溝通就行。資料成員也可以是基本型別(也就是不具備
reference 者)。如果資料成員是一個object reference,那麼你就得在某個名為建構式
(constructor)的特殊函式中(第四章討論),為該reference
進行初始化動作,藉以將它連接到某個實際物件去(一如先前所述,以
new 來執行這個動作)。如果資料成員是基本型別,你可以直接在class
的定義處直接給定初值。一如你稍後即將看到,
reference 其實也可以在定義處進行初始化。

每一\marginpar{\fbox{111}}個物件都持有用來儲存資料成員的空間;
不同的物件彼此之間並不共享資料成員。下面這個class 擁有一些資料成員:

\begin{Verbatim}[frame=single]
class DataOnly 
{
 int i;
 float f;
 boolean b;
}
\end{Verbatim}

這樣的class 什麼也不能做。不過你還是可以為它產生物件:

\begin{Verbatim}[frame=single]
DataOnly d = new DataOnly();
\end{Verbatim}

你可以指定其資料成員的值,但首先你得知道如何取得(參考到)一個物件的成員。
那就是在object reference 之後緊接著一個句點,然後再接成員名稱:

\begin{Verbatim}[frame=single]
objectReference.member
\end{Verbatim}

例如:

\begin{Verbatim}[frame=single]
d.i = 47;
d.f = 1.1f;
d.b = false;
\end{Verbatim}

當然啦,你想修改的資料也有可能位於物件所含之其他物件中。如果是這樣,
只要用句點符號把它們連接起來就行了,像這樣:

\begin{Verbatim}[frame=single]
myPlane.leftTank.capacity = 100;
\end{Verbatim}

上述的DataOnly class 除了儲存資料之外,什麼事也不能做。
因為它根本沒有任何成員函式(亦即methods)。如果想了解成員函式的運作方式,
你得先了解所謂的引數(arguments)和回傳值(return values)。
稍後對此二者將有簡短說明。
\subsubsection{基本成員(primitive members)的預設值(default values)}
當class 的某個成員屬於基本型別(primitive type)時,即使你沒有為它提供初值,
Java 仍保證它有一個預設值。表 \ref{primitive} (page \pageref{primitive})
列出各基本型別的預設值:

\marginpar{\fbox{112}}

\begin{table}[htbp]
\center
\begin{tabular}{ll}
\toprule
基本型別 & 預設值 \\
\midrule
boolean & false \\
\midrule
char & '$\backslash$u0000' (null) \\
\midrule
byte & (byte)0 \\
\midrule
short & (short)0 \\
\midrule
int & 0 \\
\midrule
long & 0L \\
\midrule
float & 0.0f \\
\midrule
double & 0.0d \\
\bottomrule
\end{tabular}
\caption{基本型別預設值}\label{primitive}
\end{table}

千萬注意,只有當變數身份是「class 內的成員」時,Java 才保證為該變數提供初值。
這能確保隸屬基本型別的那些資料成員,百分之百會有初值
(這是Java 不同於C++的地方),因而得以減少許多臭蟲發生機率。
不過語言所提供的初值對你的程式而言,或許根本牛頭不對馬嘴,甚至可能不合法。
所以最好還是養成習慣,明確為你的變數指定初值。

上述的初值保證,無法套用於區域變數(也就是並非「某個class 內的成員」)
上頭。因此如果在某個函式定義區內你這麼寫:

\begin{Verbatim}[frame=single]
int x;
\end{Verbatim}

x 便有可能是任意值(就和C/C++中一樣),不會被自動設為零。使用x
之前,你得給它一個適當值。如果忘了這麼做,Java 編譯器會在編譯時發出錯誤訊息,
告訴你該變數可能尚未初始化。這是C++不會發生的事情。
(許多C++編譯器會針對未初始化的變數給予警告,但Java 將這種情況視為一種錯誤)

\section{函式(Methods), 引數(arguments), 回傳值(return values)}
到目前為止,function(函式)這個術語用來描述某個具有名稱的子程序
(subroutine)。Java 廣泛地使用method 一詞,
表示「執行某些事情的\marginpar{\fbox{113}}方式」。
如果你希望繼續使用function 一詞,也無妨,只不過是遣詞用字上的差異。從現在起,
本書將採用method 而不採用function。(譯註: 雖然如此,但中譯本為求術語之突出,
採用「函式」一詞表示method)
Java 函式決定了某個物件究竟能夠接收什麼樣的訊息。你將在本節學到,
定義一個函式其實是很簡單的。函式基本上包括: 名稱、引數
(arguments)、回傳型別、主體。下面是最基本形式:

\begin{Verbatim}[frame=single]
returnType methodName( /* argument list */ ) 
{
 /* Method body */
}
\end{Verbatim}

函式被呼叫後,其回傳值的型別就是所謂的「回傳型別」。呼叫者希望傳
給函式的種種資訊,則化為引數列(argument list)中的型別和名稱。對
class 而言,函式的「名稱加上引數列」組合,必須獨一無二,不能重複。
(譯註:有些書籍很嚴謹區分參數parameters 和引數arguments 的不同,
本書並不如此)

Java 函式僅能做為class 的一部份。只有透過物件,而且它能夠執行某個函式,
你才能呼叫該函式\footnote{稍後你會學習所謂 static method (靜態函式),它們可透過
class(而非物件)被呼叫。}。 如果你對著某個物件呼叫了它並不具備的函式,
編譯時期就會得到錯誤訊息。如果想要呼叫某個物件的某個函式,
只要指定物件名稱,緊接著句點符號,再接續函式名稱和引數列即可,例如
objectName.methodName(arg1, arg2, arg3)。假設有個函式f(),
並不接收任何引數,回傳型別為int。如果有個名為a 的物件,允許你呼叫它所擁有的
f() 函式,那麼你可以這麼寫:

\begin{Verbatim}[frame=single]
int x = a.f();
\end{Verbatim}

其回傳值型別必須相容於x 的型別。

這種「呼叫函式」的行為,稱為「發送訊息給物件」。上述例子中,a 是物件,
f() 相當於訊息。物件導向程式設計常常被簡單地歸納為「將訊息發送給物件」的方式。

\subsection{引數列(The argument list)}\marginpar{\fbox{114}}
外界傳給函式的資訊,由引數列指定。就如你所想的一樣,這些資訊和
Java 的其他資訊一樣,都以物件的形式出現。
所以引數列中必須指定每一個想要傳入的物件的型別和名稱。
Java 之中,所有傳遞物件的場合(包括現在所討論的這個),傳遞的都是物件的
reference\footnote{先前所提的那些「特殊」資料型別:boolean、char、byte、short、
int、long、 float、double,都是這句話的例外。一般來說,傳遞物件,
其實就是傳遞物件的 reference。},其型別必須正確。
如果引數為String,那麼你傳入的就必須是個 String 物件才行。

假設有個函式,接受String 為其引數。底下是其定義。這份定義必須被置放於
class 定義式中,才能夠被正確編譯:

\begin{Verbatim}[frame=single]
int storage(String s) 
{
 return s.length() * 2;
}
\end{Verbatim}

這個函式告訴你,如果要儲存指定的String 物件,需動用多少bytes。為了支援
Unicode 字元集,Java 字串的每個字元都是16 bits 或說兩個
bytes。引數型別是String,名稱是s。當s 被傳入此一函式,
它和其他物件就沒有什麼兩樣了(甚至你可以發送訊息給它)。本例呼叫了s 的
length(),那是String 提供的眾多函式之一,會回傳字串中的字元數。

你看到了,上述例子使用關鍵字return 做兩件事情。首先,它代表「離開這個函式」,
意謂事情做完了。其次,如果執行過程中誕生了回傳值, 這個回傳值應該擺在
return 之後。本例之中,回傳值係透過s.length() * 2 這個式子獲得。

定義函式時,你可以決定任何你想要回傳的型別。但如果這個函式並不打算回傳任何東西,
你應該將回傳型別指定為 void,例如:

\begin{Verbatim}[frame=single]
boolean flag() { return true; }
float naturalLogBase() { return 2.718f; }
void nothing() { return; }
void nothing2() {}
\end{Verbatim}

\marginpar{\fbox{115}}

當回傳型別為void 時,關鍵字return 就只是用來離開函式。
我們不必等到執行至函式最末端才離開,可以在任意地點回返。
但如果函式的回傳型別並非
void,那麼不論自何處離開,編譯器都會要求你一定得回傳適當型別的回傳值。

閱讀至此,你或許會覺得,程式看起來似乎像是許多「帶有函式」之物件的組合。
這些函式可以接受其他物件作為引數,也可以發送訊息給其他物件。
以上說法的確還有許多需要補充,不過接下來的章節中,
我要先告訴你如何在函式中做判斷,以便執行更細膩的動作。對本章而言,
「發送訊息」的討論已經夠了。
\section{打造一個Java 程式}
在看到你的第一個Java 程式之前,還有一些主題是你必須了解的。
\subsection{名稱的可視性(Name visibility)}

「名稱管理」對所有程式語言而言,都是個重要課題。
如果你在程式的某個模組中使用了某個名稱,
另一位程式員在同一程式的另一個模組中使用同樣的名稱,
該如何區分二者才能使它們不牴觸呢?C 裡頭的這個問題格外嚴重,
因為程式之中往往充滿許多難以管理的名稱。C++ classes(Java classes 的師法對象)
將函式包裝於內,這麼一來就可以和其他classes 內的同名函式隔離,
避免名稱衝突的問題。不過C++ 仍允許全域資料(global data)和全域函式
(global functions)的存在,所以還是有可能發生命名衝突。為了解決這個問題,
C++透過了幾個關鍵字,引入所謂的「命名空間(namesapces)」概念。

Java 採用一種全新方法,避免上述所有問題。
為了讓程式庫內的名稱不致於和其他名稱相混,Java 採用和Internet 域名
(domain names)相似的指定詞,進一步規範所有名稱。事實上,Java 創設者希望你將
Internet 域名反過來寫,確保所有名稱在這個世界上都是獨一無二的。
我的域名是\marginpar{\fbox{116}}
BruceEckel.com,所以我的一些奇奇怪怪的做為工具之用的程式庫,就命名為
com.bruceeckel.utility.foibles。當你將域名反轉,
其中的句點便用來代表子目錄的劃分。
(譯註:例如com.bruceeckel.utility.foibles 便置於
com/bruceeckel/utility/foibles 目錄下。在這裡,作者探討的是Java
裡頭的package 命名方式,但未明示package 這個字眼,所以特此提醒讀者。)

在Java 1.0 和1.1 中,域名最末的com、edu、org、net 等等,按慣例都應該大寫,
所以上例應該寫成: COM.bruceeckel.utility.foibles。Java 2 發展到半途的時候,
發現這麼做會引起一些問題,因此,現在package 的整個名稱都是小寫了。

上述機制意謂你的檔案都能夠存在於它們自有的命名空間中(譯註:每個
package 都是一個獨一無二的命名空間),而同一個檔案中的每個
class 都得有個獨一無二的識別名稱。這麼一來就不需特意學習其他語言的什麼功能,
便能夠解決這個問題。是的,程式語言自動為你處理掉這個問題。
\subsection{使用其他組件(components)}
當你想在程式中使用事先定義好的classes 時,編譯器必須知道它們的位置。當然,
它們可能位於同一個原始碼檔案中,那就可以直接在這個檔案中呼叫。這種情形下,
你只管使用這個class - 即使它的定義在檔案稍後才出現。
Java 解決了「前置參考(forward referencing)」問題,所以你完全不必傷腦筋。

如果classes 位於其他檔案之中,又該如何?
你可能會認為編譯器應該有足夠的聰明找到那個位置,可惜事實不然。
想像你正要使用某個具有特定名稱的 class,但它卻有好幾份定義
(假設各不相同)。或者更糟的是,想像你正在撰寫某個程式,開發過程中你將某個新
class 加到程式庫中,因而和某個舊有的class 發生了命名衝突。

為了解決這種問題,你得消除所有可能發生的混淆情形。關鍵字import 可以明確告訴
Java 編譯器,你想使用的class 究竟是哪一個,以便消除所有可能發生的混淆。
import 能夠告訴編譯器引入哪一個package - 那是由classes 組成的一個程式庫。
(在其他語言中,程式庫不僅內含\marginpar{\fbox{117}}
classes,也可以內含函式和資料,但是在Java 裡頭,所有程式碼都必須寫在
class 內。)

大部份時候,你會使用Java 標準程式庫內的種種組件。
這個程式庫是和編譯器附在一起的。使用這些組件時,你並不需要寫上一長串的反轉域名。
舉個例子,你只要這麼寫就行了:

\begin{Verbatim}[frame=single]
import java.util.ArrayList;
\end{Verbatim}

這便是告訴編譯器說,你想使用Java 的ArrayList class。如果你還想使用
util 內的其他classes,又不想逐一宣告,那就只要以* 號代表即可:

\begin{Verbatim}[frame=single]
import java.util.*;
\end{Verbatim}

一次宣告一大群classes,比個別匯入(import)一個個classes,常見且方便多了。
(譯註:雖然比較方便,卻會影響編譯時間)
\subsection{關鍵字 static}
一般而言,當你設計某個class 時,其實就是在描述其物件的外觀長相及行為舉措。
除非以new 來產生物件,否則並不存在任何實質的物件。產生物件之際,
儲存空間才會配置出來,其函式才可供外界使用。

但是有兩種情況,是上述方式所無法解決的。第一種是,你希望不論產生了多少個物件,
或不存在任何物件的情形下,那些特定資料的儲存空間都只有一份。第二種情況是,
你希望某個函式不要和class object 綁在一起。
透過關鍵字static,便可以處理這兩種情況。當你將某筆資料成員或某個函式宣告為
static,它就不再被侷限於所屬的class object 上。所以,即使沒有產生任何
class object,外界還是可以呼叫其static 函式,或是取用其
static data。一般情形下,你得產生某個物件,再透過該物件取用其資料和函式。所以,
non-static 資料/函式必須知道它們隸屬於哪一個物件,才有辦法運作。由於使用
static 函式前並不需要先產生任何物件,所以在
static 函式中不能「直接」取用non-static 資料/函式。如果只是單純地直接呼叫
non-static 函式,而沒有指定某個物件,是行不通的,原因是
non-static 資料/函式總是得和特定的物件綑綁在一起。

某些物件導向程式語言,以\marginpar{\fbox{118}}
class data 和class methods 兩個詞彙,代表那些不和特定物件有所關聯,
「其存在只是為了class」的資料和函式。有時候 Java 相關文獻也會使用這兩個詞彙。

只要將關鍵字static 擺在資料成員或函式的定義前,就可以使它們成為靜態。
以下便可以產生一筆static 資料成員,並將它初始化:

\begin{Verbatim}[frame=single]
class StaticTest 
{
 static int i = 47;
}
\end{Verbatim}

現在,即使你產生兩個StaticTest 物件,StaticTest.i 仍然只有一份。
產生出來的那兩個物件會共用同一個 i。再看這個:

\begin{Verbatim}[frame=single]
StaticTest st1 = new StaticTest();
StaticTest st2 = new StaticTest();
\end{Verbatim}

此時st1.i 和st2.i 的值一樣,都是47,因為它們都指向同一塊記憶體。

有兩種方法可以取用宣告為static 的變數。一如上例,你可以透過某個物件來定址,例如
st2.i。也可以直接經由其class 名稱完成參考動作 - 這種作法對
non-static 成員是行不通的(但對於static 成員卻比較好,
因為這種寫法可以更強調所參考的對象是個static 成員)。

\begin{Verbatim}[frame=single]
StaticTest.i++;
\end{Verbatim}

++運算子會將變數值累加1。經過這個動作,st1.i 和st2.i 的值都是48。

相同的模式可以推廣到static 函式。你可以透過物件來取用某個static 函式,
和取用其他普通的函式沒什麼兩樣。你也可以經由以下特殊語法取用之:
ClassName.method()。定義static 函式的方式和定義static data member
的方式十分類似:

\begin{Verbatim}[frame=single]
class StaticFun 
{
 static void incr() { StaticTest.i++; }
}
\end{Verbatim}

你看到了,StaticFun incr() 將static 資料成員i 的值累加一。你可以用一般方式,
透過物件來呼叫incr():

\begin{Verbatim}[frame=single]
StaticFun sf = new StaticFun();
sf.incr();
\end{Verbatim}
\marginpar{\fbox{119}}

但由於incr() 是static 函式,所以你也可以直接透過class 加以呼叫:
\begin{Verbatim}[frame=single]
StaticFun.incr();
\end{Verbatim}

某個資料成員被宣告為static 之後,勢必會改變其建立方式(因為static
資料成員對每個class 而言都只有一份,而non-static 資料成員則是每個物件各有一份)。
對static 函式來說,差別反而沒那麼大。Static 函式的最重要用途之一,
就是讓你可以在不建立任何物件的情形下,還可以呼叫之。這一點很重要,
稍後我們會看到,我們得透過main() 的定義,做為程式的執行起點。

就像任何函式一樣,static 函式可以產生或使用其型別所衍生的具名物件,所以
static 函式常常被拿來當作「牧羊人」的角色,
負責看管眾多隸屬同一型別的一整群物件。
\section{初試啼聲你的第一個Java 程式}
終於要寫一個真正的程式了\footnote{某些開發環境
(programming environments)會將程式輸出畫面快速捲過,在你看不清楚任何東西之前,
就什麼都沒有了。你可以將下列一小段程式碼加到
main()的最末端,使輸出結果在程式結束前暫停一下:

%\begin{Verbatim}[frame=single]
\noindent
try \\
$\{$\\
 System.in.read();\\
$\}$\\
catch(Exception e) $\{$ $\}$ \\
%\end{Verbatim}

這麼做便能凍結輸出結果,直到按下 ``Enter'' 或其他鍵。以上程式碼所用到的觀念,
一直要到本書後段才會出場。所以此刻你可能無法了解它的作為,
但至少它能派上用場。}。此程式一開始會印出一個字串,然後利用
Java 標準程式庫中的Date class 印出日期。請注意其中運用了一種新的註解方式,
雙斜線 //,在它之後出現的所有直到行末為止的內容,都會被編譯器視為註解。

\begin{Verbatim}[frame=single]
// HelloDate.java
import java.util.*;
public class HelloDate 
{
 public static void main(String[] args) 
 {
  System.out.println("Hello, it's: ");
  System.out.println(new Date());
 }
}
\end{Verbatim}
\marginpar{\fbox{120}}

你得將import 敘述句置於每個程式檔的起始處,藉以將該檔案所需要的所有額外的
classes 含括進來。請注意我說「額外的」,因為有一組
classes 會被自動含括於每個Java 程式檔中,我說的是 java.lang。請開啟你的
Web 瀏覽器,並檢閱Sun 的公開文件 (如果你尚未從 java.sun.com 下載這份文件,
或尚未以其他方式安裝Java 文件的話,此其時矣)。你可以在各個
packages 的列表上找到所有Java 標準程式庫。請點選
java.lang,畫面上會出現一份列表,顯示出其中的所有classes。
由於 java.lang 會被自動含括於每個Java 程式檔中,所以這個程式庫內的所有
classes 都不需要做import 宣告,便可直接運用。不過 java.lang 列表中並沒有
Date class,這表示你必須匯入(import)另一個程式庫才能夠使用它。如果你不知道某個
class 位於哪一個程式庫內,或是如果你想同時瀏覽所有classes,你可以在
Java 文件中選擇"Tree",然後便可看到隨Java 而附的所有
classes。請使用瀏覽器的 ``find'' 功能搜尋Date。如果你照著上述步驟,
就可以看到這個class 列於 java.util.Date,現在你知道它屬於
util 程式庫了。為了使用Date class,你可以寫 import java.util.* 將它括進來
(譯註:或標明java.util.Date 也行)。

現在回到最開始的地方,選擇 java.lang 和System,你會看到System class 有許多欄位。如果你再選擇out,就可以看到它其實是個static PrintStream 物件。因為它是靜態的,
所以你不必做任何事情,out 物件便已存在,你只管使用便是。我們到底能對
out 物件做什麼事,取決於其型別: PrintStream。Java 這份文件設計得很方便,
你所看到的 PrintStream 是個超鏈結( hyperlink ) , 只要點選它, 便可看到
PrintStream 提供給外界呼叫的所有函式,數量相當多,本書稍後會加以討論。
此刻我們只對println() 感興趣,它的實際作用對我們而言是:
「印出我要你印在螢幕上的東西,完成後換行」。因此在你撰寫的Java 程式中,
當你想要將某些訊息列印到螢幕上, 可使用
System.out.println("things") 完成。

class\marginpar{\fbox{121}}
名稱必須與檔案主檔名相同。你所開發的程式如果和目前這個程式一樣,
是個獨立執行的程式,那麼檔案中必須有某個class,名稱和檔案主檔名相同。
否則編譯器會顯示錯誤訊息。那個class 必須含有一個main(),
其標記式(signature)必須是:

\begin{Verbatim}[frame=single]
public static void main(String[] args) {
\end{Verbatim}

其中的關鍵字public,表示這是一個要公開給外界使用的函式(第五章有更詳細的說明)。
傳入main() 的引數,是個String 物件陣列。這個程式並未使用args,但
Java 編譯器會嚴格要求你一定要這麼宣告main(),
因為args 被拿來儲存「命令行(command line)引數」。
印出日期的這一行程式碼,相當有趣:

\begin{Verbatim}[frame=single]
System.out.println(new Date());
\end{Verbatim}

看看傳入的引數:首先產生一個Date 物件,然後直接傳給println()。
當這個敘述句執行完畢,產生出來的Date 物件再也不會被使用了,因此垃圾回收器
(garbage collector)便會在適當時機將這個物件所佔據的空間收回。
此一物件的清除事宜,我們一點也不用擔心。

\subsection{編譯與執行 (Compiling and running)}

想要編譯並執行這個程式,以及本書的所有其他程式,你必須先將
Java 開發環境安置妥當才行。目前有許多協力廠商或團體推出各種開發環境,
本書假設你使用Sun 釋出的免費JDK。如果你使用其他開發系統,
你可能需要好好調閱其上所附的文件,決定如何編譯和執行。

上網,然後連到 java.sun.com。你可以在這個網站找到相關訊息和鏈結,
這些資訊和鏈結會引導你下載符合你的硬體平台的JDK,並加以安裝。

一旦JDK 安裝妥當,你也設定好你的路徑(path),使電腦能夠找到
javac 和java 這兩個可執行檔(譯註:請參考p.xxxiii 之「Java 環境設定」),
請下載本書原始程式碼(也可以在本書所附光碟中找到,或者從
www.BruceEckel.com 網站取得)。解壓縮之後會自動依本書章節,建立起不同的子目錄。
現在,請移駕至子目錄 c02,鍵入:

\begin{Verbatim}[frame=single]
javac HelloDate.java
\end{Verbatim}
\marginpar{\fbox{122}}

這一行命令應該不會產生任何回應。如果你發現任何錯誤訊息,便表示你沒有妥當安裝好
JDK,你得回頭檢查問題出在哪裡。
如果你沒有得到任何回應訊息,請接著輸入:

\begin{Verbatim}[frame=single]
java HelloDate
\end{Verbatim}

然後便能夠看到程式中的訊息和當天的日期被輸出於螢幕上。
以上整個過程同時也是本書每一個程式的編譯和執行過程。
你還會看到本書所附的原始碼中,每章都有一個名為makefile 的檔案,
這個檔案是提供給 ``make'' 指令用的,可以自動造出(build) 該章的所有檔案。
www.BruceEckel.com 網站上的本書相關網頁,可以告訴你更多如何使用
makefile 的詳細資訊。

\section{註解及內嵌式文件 \\
Comments and embedded documentation}

Java 提供兩種註解風格。一種是傳統C 所用的註解風格,C++也承繼了它。
此種註解以 /* 為首,後接的註解內容可能跨越多行,最後以*/ 結尾。請注意,
許多程式員喜歡在多行註解中以 * 做為每行起頭,所以常常可以看到這樣子的寫法:

\begin{Verbatim}[frame=single]
/* This is a comment
* that continues
* across lines
*/
\end{Verbatim}


記住,\marginpar{\fbox{123}}
/* 和*/ 之間的所有內容,都會被編譯器忽略,所以上述寫法和以下寫法沒有什麼兩樣:

\begin{Verbatim}[frame=single]
/* This is a comment that
continues across lines */
\end{Verbatim}

Java 的第二種註解風格源於C++。這種註解用於單行,以// 為首,直至行末。
這種註解十分方便,並且因為它的易用性而被廣泛使用。有了這種註解方式,
就不必在鍵盤上辛苦尋找 / 和 * 的位置,只需連按兩次相同鍵即可,
而且不必注意註解符號的成對問題。所以你常會看到這種寫法:

\begin{Verbatim}[frame=single]
// this is a one-line comment
\end{Verbatim}

\subsection{寓文件於註解 \\
Comment documentation}
Java 語言有一項經過深思熟慮之後才有的設計。
Java 設計者不認為程式碼的撰寫是唯一重要的工作 -
他們認為說明文件的重要性不亞於程式碼本身。在程式碼說明文件的撰寫上,
最難的問題大概就是文件本身的維護了。如果文件和程式碼二者分離,那麼,
每次改變程式碼就得一併更動文件,會是一件很麻煩的事。
這個問題的解決辦法似乎很單純:讓程式碼和文件鏈結在一起。想要達到這個目的,
最簡單的作法就是把它們都擺入同一個檔案中。不過,更精確地說,
這種作法需要某種特殊語法,用以標示出「和程式碼融合在一起」的文件形式。
此外也需要一個工具,將這些註解文件自程式碼檔案中提解出來,
轉換為實際可用之文件形式。這正是 Java 的作法。

javadoc 就是用來將程式碼內嵌文件提解出來的工具。這個工具使用Java
編譯器的某些技術,藉以搜尋比對程式檔中的註解。這個工具不僅會解出由特定標籤
(tags)所標示的資訊,也會擷取出這些資訊所屬的class 名稱或函式名稱。如此一來,
一份端端正正、合乎法度的文件,便可以在最小力氣下產生出來。

javadoc 的輸出結果是HTML 檔案,透過Web 瀏覽器便可閱讀。
這個工具讓你只需維護程式檔, 便可自動產生出立即可用的文件。
有了\marginpar{\fbox{124}}
javadoc,大家都可以在文件產生的標準上有所依歸。我們也可以期望或甚至要求,所有
Java 程式庫都得提供相關的說明文件。

\subsection{語法}

所有的javadoc 命令句,都必須置於以/** 為首的註解內,並以*/ 作為結束。
Javadoc 的運用有兩種主要型式:內嵌式HTML 或文件標籤(doc tags)。
所謂文件標籤是一種以@ 符號為首的命令,這個符號必須置於註解的最前面,
也就是扣掉最前面的 * 之後的最前面處。

文件內容一共有三種型式,分別對應於註解位置前的三種元素:class、
variable、method。換句話說,class 的註解必須恰好出現在class 定義式之前;
variable(變數)的註解必須恰好出現在變數的定義之前;method
(函式)的註解必須恰好出現在函式的定義式之前。以下是個簡單範例:

\begin{Verbatim}[frame=single]
/** A class comment */
public class docTest 
{
 /** A variable comment */
 public int i;
 /** A method comment */
 public void f() {}
}
\end{Verbatim}

請註意,javadoc 只會針對public 或protected 成員,進行註解文件的處理。
private 或所謂的friendly 成員(請參閱第五章)會被略去,輸出結果中看不到它們
(不過你也可以打開-private 選項,使宣告為private 的成員一併被處理)。
這麼做是有道理的,因為只有宣告為public 和 protected 的成員,才能夠為外界所取用;
用戶端(client)也只能看到這些。至於針對 class 所做的註解,都會被處理並輸出。

上述程式碼的輸出結果是個HTML 檔,其格式就和所有Java 文件所遵循的標準一樣,
對任何使用者而言肯定不陌生,可從其中輕鬆觀看你所產生的classes 內容。
當你逐一輸入上述程式碼,然後透過javadoc 產生HTML 文件,
最後再以瀏覽器觀看產出結果,你會覺得這麼做非常值得。

\subsection{內嵌的HTML} \marginpar{\fbox{125}}
javadoc 能夠將你所設定的HTML 控制命令,加到它所產生的HTML 文件中。
這個功能讓你可以盡情發揮HTML 的功能。不過最主要的目的還是讓你用於程式碼的編排,
例如:

\begin{Verbatim}[frame=single]
/**
* <pre>
* System.out.println(new Date());
* </pre>
*/
\end{Verbatim}

你也可以像編修其它網頁一樣,使用HTML 格式,將你所製作的一般文字描述,
加以排列美化:

\begin{Verbatim}[frame=single]
/**
* You can <em>even</em> insert a list:
* <ol>
* <li> Item one
* <li> Item two
* <li> Item three
* </ol>
*/
\end{Verbatim}

注意,註解文件中每一行最前面的星號和空白字元,會被javadoc 忽略。
javadoc 會將所有內容重新編排過,俾使輸出結果能夠符合標準的文件外觀規範。請
千萬不要在內嵌的HTML 中使用諸如$<$h1$>$或$<$hr$>$之類的標題標籤(headings),因為
javadoc 會插入自己的標題標籤,你的標題會對它造成干擾。

不論是classes、variables、methods 的註解說明,都支援這種內嵌HTML
的功能。

\subsection{@see: 參考其他classes}
三種不同型態(classes、variables、methods)的註解說明中,都可以含入
@see 標籤。這個標籤的功能是讓你得以參考其他class 的說明文件。
javadoc 會自動為@see 標籤產生一個HTML 超鏈結(hyperlinks),鏈結到其他文件。
格式如下:
\marginpar{\fbox{126}}
\begin{Verbatim}[frame=single]
@see classname
@see fully-qualified-classname
@see fully-qualified-classname#method-name
\end{Verbatim}
這三種格式都會使產出的文件中多出「See Also」超鏈結。javadoc
並不檢查你所提供的超鏈結內容是否存在,所以你得自己注意。
\subsection{Class(類別)文件所用的標籤}
除了內嵌HTML 和@see 標籤之外,針對類別(classes)而做的文件說明,
還可以使用其他標籤來標示作者和版本資訊。這些也都可以套用在
interfaces(見第八章)身上。
\subsubsection{@version}
格式如下:
\begin{Verbatim}[frame=single]
@version version-information
\end{Verbatim}
其中的version-information(版本資訊)可以是你認為需要加進去的任何重要訊息。不過,
只有在執行javadoc 時將-version 選項打開,上述的版本資訊才會被加到所產生的
HTML 文件內。
\subsubsection{@author}
格式如下:
\begin{Verbatim}[frame=single]
@author author-information
\end{Verbatim}
其中的author-information(作者資訊),顧名生義,就是你的名字。
也可以包含你的電子郵箱或其他適合加入的資訊。不過,只有在執行
javadoc 時將-author 選項打開,上述的作者資訊才會被加到所產生的
HTML 文件內。

你也可以提供多個author 標籤,列出所有作者。這些author 標籤必須連續出現。產出的
HTML 中,所有作者資訊都會被併到同一段。

\marginpar{\fbox{127}}

\subsubsection{@since}
這個標籤讓你指定程式碼所使用的最早版本。你可以在HTML Java 文件中,
看到這個標籤被用來指出所使用的JDK 版本。

\subsection{Variable(變數)文件所用的標籤}
變數說明文件中,除了@see 標籤外,只能使用內嵌HTML 的方式。
\subsection{Method(函式)文件所用的標籤}
除了@see 標籤和內嵌HTML 的方式外,針對函式所做的說明文件,
還可以使用描述其參數、回傳值、異常(exceptions)等的控制標籤。
\subsubsection{@param}
格式如下:
\begin{Verbatim}[frame=single]
@param parameter-name description
\end{Verbatim}
其中的parameter-name(參數名稱)是位在參數列中的識別字,
description(描述句)則是純粹文字,可以延續數行,直到遇上新標籤才結束。
此一標籤的使用次數不限,通常我們會在撰寫時為每一個參數提供一份說明。
\subsubsection{@return}
格式如下:
\begin{Verbatim}[frame=single]
@return description
\end{Verbatim}
其中的description 用來描述回傳值的意義;可以延續數行。
\subsubsection{@throws}
本書第10 章才會討論異常(exception)。簡單地說,那是一種可以被
「擲出(throw)」的物件,當函式在執行過程中發生問題,可以擲出異常,
告訴外界發生了這個錯誤。當你呼叫某個函式時,
雖然最多只可能出\marginpar{\fbox{128}}現一個異常,
但函式可能產生任意多個不同型別的異常物件,它們都需要加以描述。
描述異常用的標籤格式是:
\begin{Verbatim}[frame=single]
@throws fully-qualified-class-name description
\end{Verbatim}
其中的full-qualified-class-name 代表某個異常類別 (exception class)
的完整名稱,必須獨一無二(譯註:所謂完整路徑是指包含
package 名稱)。description 則是用來說明在什麼情形下呼叫這個函式,
會產生此一型別的異常。
\subsubsection{@deprecated}
這個標籤標示,此一函式已被更新的功能替換,建議使用者不要再使用它,
因為它可能在不久的將來被移除。如果你在程式中運用被標示為
@deprecated 的函式,編譯器會發出警告。
subsection{文件製作實例}
底下這個程式,是先前介紹過的第一個 Java 程式。這次加上了註解命令:
\begin{Verbatim}[frame=single]
//: c02:HelloDate.java
import java.util.*;
/** The first Thinking in Java example program.
* Displays a string and today's date.
* @author Bruce Eckel
* @author www.BruceEckel.com
* @version 2.0
*/
public class HelloDate 
{
 /** Sole entry point to class & application
 * @param args array of string arguments
 * @return No return value
 * @exception exceptions No exceptions thrown
 */
 public static void main(String[] args) 
 {
  System.out.println("Hello, it's: ");
  System.out.println(new Date());
 }
} ///:~
\end{Verbatim}
\marginpar{\fbox{129}}

在第一行中,我使用自己獨特的方式,將 ``:'' 作為一個特殊標示,
描述此一註解所在之原始碼檔案的名稱。這一行包含了檔案的完整路徑(本例的
c02 代表第 2 章),然後是檔案名稱\footnote{我利用 Python
(請參考www.Python.org)寫出一個工具,能夠根據這份資訊萃取出程式檔,
擺放在適當的子目錄下,並產生makefile。}, 最後一行也以註解作收,
代表這份原始碼已經到了盡頭,此後再無內容。
如此一來便能夠自動化地將這些程式碼自本書文字中取出,再以編譯器驗證之。
\section{撰碼風格(Coding style)}
Java 對於classes(類別)的命名有一個不成文規定:名稱的第一個字母大寫。
如果名稱之中含有許多個別字,就把這些字併在一塊兒(不以底線連接),
每一個被併在一塊兒的字的第一個字母都採大寫。例如:

\begin{Verbatim}[frame=single]
class AllTheColorsOfTheRainbow { // ...
\end{Verbatim}

幾乎所有名稱,包括methods(函式)、fields(資料成員)、object reference,
命名方式都遵循上述法則,唯一的例外是,它們的名稱第一個字母採用小寫而非大寫。
例如:
\begin{Verbatim}[frame=single]
class AllTheColorsOfTheRainbow 
{
 int anIntegerRepresentingColors;
 void changeTheHueOfTheColor(int newHue) 
 {
  // ...
 }
 // ...
}
\end{Verbatim}
當然,你應該體認,classes 使用者在撰寫程式時,也得輸入這麼長的名字。
所以命名時請多為使用者著想,別過度使用無意義又冗長的名稱。

Sun 程式庫中的Java 程式碼,其左大括號和右大括號的擺放方式,和本書的風格一致。
\marginpar{\fbox{130}}
\section{摘要}
你已經從本章中看到撰寫一個簡單Java 程式所應具備的相關知識。你也已經對
Java 程式語言有了一個概括性的認識,並了解某些Java 基本觀念。
到目前為止,所有範例都只停留在「做這件事情,然後做這件事情,接著做那件事情...」
的形式。如果你希望程式能夠進行一些判斷,像是:「如果做這件事情的結果是紅色,
那麼就做那件事情;否則,就做另一件事情...」,這該怎麼進行呢?
Java 所提供的此類程式編寫基本行為,將在下一章說明。
\section{練習}
\footnotesize
某些經過挑選的題目,其解答置於
《The Thinking in Java Annotated Solution Guide》電子文件中。
僅需小額費用便可自www.BruceEckel.com.網站取得。

\normalsize
\begin{enumerate}
\item 請仿照本章的HelloDate.java 範例程式,寫一個單純印出「hello,
world」的程式。你所撰寫的class 只需具備一個函式即可,
也就是程式執行時第一個執行起來的main()。請不要忘了將它宣告為
static,並為它指定引數列- 即使你不會用到這個引數列。以
javac 編譯這個程式,並以java 執行它。如果你的開發環境不是
JDK,請學習如何在你的環境下進行程式的編譯和執行。
\item 找出含有ATypeName 的程式片段,將它變為一個完整的程式,
編譯後執行。
\item 將DataOnly 的程式片段變成一個完整程式,編譯後執行。
\item 修改習題3 所完成的程式,讓DataOnly 中的資料可以在main()
中被指定,並列印出來。
\item 撰寫某個程式,含入本章所定義的storage(),並呼叫之。
\item 將 StaticFun 程式片段,轉變成一個整個的程式,編譯後執行。

\item 撰寫某個程式,\marginpar{\fbox{131}}使它能夠接受由命令列
(command line)傳入的三個引數。為此,
你得對代表命令列引數的Strings 陣列進行索引。
\item 將 AllTheColorsOfTheRainbow 這個範例改寫成真正程式,編譯後執行。
\item 找出HelloDate.java 的第二個版本,也就是本章用來說明註解文件的那個程式例。
請對該檔案執行javadoc,並透過你的瀏覽器觀看產生的結果。
\item 將docTest 儲存為檔案並編譯之。然後使用javadoc 為它產生文件。最後,
透過你的瀏覽器觀看產生的結果。
\item 在練習10 中,以內嵌HTML 的方式加入一個HTML 項目列表。
\item 使用練習1 的程式,為它加上註解文件。使用javadoc 為此註解文件產生一個
HTML 檔,並以你的瀏覽器觀看產生的結果。

\end{enumerate}
